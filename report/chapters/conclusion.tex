%
% The MIT License (MIT)
%
% Copyright (c) 2016 Paul Batty
%
% Permission is hereby granted, free of charge, to any person obtaining a copy
% of this software and associated documentation files (the "Software"), to deal
% in the Software without restriction, including without limitation the rights
% to use, copy, modify, merge, publish, distribute, sublicense, and/or sell
% copies of the Software, and to permit persons to whom the Software is
% furnished to do so, subject to the following conditions:
%
% The above copyright notice and this permission notice shall be included in
% all copies or substantial portions of the Software.
%
% THE SOFTWARE IS PROVIDED "AS IS", WITHOUT WARRANTY OF ANY KIND, EXPRESS OR
% IMPLIED, INCLUDING BUT NOT LIMITED TO THE WARRANTIES OF MERCHANTABILITY,
% FITNESS FOR A PARTICULAR PURPOSE AND NONINFRINGEMENT. IN NO EVENT SHALL THE
% AUTHORS OR COPYRIGHT HOLDERS BE LIABLE FOR ANY CLAIM, DAMAGES OR OTHER
% LIABILITY, WHETHER IN AN ACTION OF CONTRACT, TORT OR OTHERWISE, ARISING FROM,
% OUT OF OR IN CONNECTION WITH THE SOFTWARE OR THE USE OR OTHER DEALINGS IN
% THE SOFTWARE.
%

\section{Conclusion}
\label{sec:conclusion}

The start of this paper defined some very clear goals that this project hoped to achieve by the end of it. Firstly, to understand of why SQLite is so good, what makes it so prevalent and how it works. This included the file format and its systems. This was achieved throughout the first chapter of this paper. 
\\\\
The second aim was to take this knowledge and build a tool that could record all operations performed onto the database. While providing the same insight gathered throughout this project without having to look through a Hex editor. It should also be easy to use and well tested. Making it reliable and efficient. This was successfully achieved through the second to forth sections.
\\\\
Lastly, to look at where this project could be taken in the future, and what could be done to take the application to the next stage. This involved critically evaluating the final application and what could be changed or added. This was achieved in the final sections of this paper.
\\\\
In conclusion this project has been an overall success. The main aims that have been set out were reached. However, the performance could still be improved. The user interface stills need some polishing in order to make it more user friendly. Providing a better visualisation of the database. Though the last stretch is always the longest and a lot of time could be spent polishing the interface, and fixing all the edge cases that have not yet made themselves apparent.
\\\\
In the future, other features could be added such as providing support for the other system such as  extensions, lock byte and pointer map pages, and any other changes made to SQLite. Another project that would also be useful stemming off of this one is to try and recreate the original SQL query sent to the database based on the changes made, since it can only currently only list the changes. 
\\\\
But, with that said this has been an enjoyable project, and by then end of it all have learnt a great deal about SQLite, Java and JavaFX, some more of the unique data structures such as the Merkle trees. In addition to a nice tool that can be used in the future whenever working on a SQLite database to discover any problems.