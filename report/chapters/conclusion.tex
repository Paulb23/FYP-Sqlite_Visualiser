\section{Conclusion}
\label{sec:conclusion}

At the start of this paper, I defined some very clear goals that I hoped to achieve by the end of it. Firstly, I wanted to extend my knowledge and understand why SQLite is so good, what makes it so prevalent and how it works. This included the file format and its systems. This was achieved throughout the first chapter of this paper. 
\\\\
The second aim was to take this knowledge and build a tool that could record all operations performed onto the database. And provide the same insight I have gathered throughout this project without having to look through a Hex editor. It should also be easy to use and well tested. Making it reliable and efficient. This was successful achieved through the send to fifth sections.
\\\\
In conclusion this project has been an overall success. The main aims that i have set out have all been met. However, the performance could still be improved. The user interface stills need some polishing in order to make it more user friendly. and provide a better visualisation of the database. Though the last stretch is always the longest and I could spend a very long time polishing the  interface, and fixing all the edge cases that have not yet made themselves apparent. In the future, I would like to proved support for the other system such as  extensions, lock byte and pointer map pages, and any other changes made to SQLite. Anther project that would also be useful stemming off of this one is to try and recreate the original SQL query sent to the database based on the changes made, since it can only currently only list the changes. 
\\\\
But, with that said this has been an enjoyable project, and by then end of it all I have learnt a great deal about SQLite, Java and JavaFX, some more of the unique data structures such as the Merkle trees. And I have got a nice tool that can be used in the future, whenever I am working on a SQLite database to discover any problems.