%
% The MIT License (MIT)
%
% Copyright (c) 2014 Paul Batty
%
% Permission is hereby granted, free of charge, to any person obtaining a copy
% of this software and associated documentation files (the "Software"), to deal
% in the Software without restriction, including without limitation the rights
% to use, copy, modify, merge, publish, distribute, sublicense, and/or sell
% copies of the Software, and to permit persons to whom the Software is
% furnished to do so, subject to the following conditions:
%
% The above copyright notice and this permission notice shall be included in
% all copies or substantial portions of the Software.
%
% THE SOFTWARE IS PROVIDED "AS IS", WITHOUT WARRANTY OF ANY KIND, EXPRESS OR
% IMPLIED, INCLUDING BUT NOT LIMITED TO THE WARRANTIES OF MERCHANTABILITY,
% FITNESS FOR A PARTICULAR PURPOSE AND NONINFRINGEMENT. IN NO EVENT SHALL THE
% AUTHORS OR COPYRIGHT HOLDERS BE LIABLE FOR ANY CLAIM, DAMAGES OR OTHER
% LIABILITY, WHETHER IN AN ACTION OF CONTRACT, TORT OR OTHERWISE, ARISING FROM,
% OUT OF OR IN CONNECTION WITH THE SOFTWARE OR THE USE OR OTHER DEALINGS IN
% THE SOFTWARE.
%

\section{Testing}
\label{sec:testing}

 I mentioned at the start of this paper I wanted to make sure that my tool could be relied on. In order to accomplish this I used a variety of testing methods and tool that we will go over in this section.

\subsection{Test data}
\label{subsec:test_data}

In order to assure that my program ran correctly under a variety of circumstances, I used a variety of different sized databases. The smallest made up of one table and page, and the largest twelve tables and 1066 pages. 

\subsection{Unit tests}
\label{subsec:unit_tests}

Throughout the implementation stage I kept close to test driven development, and as such have written a lot of unit tests utilising the JUnit framework by \cite{junit} . In total I have around 120 unit tests. All passing. The unit tests are written to use mocks where dependences are needed allowing me to make sure it is working as intended..
\\\\
In order to test the user interface I found a test framework that works alongside JUnit called TestFX \citep{test_fx} it is specifically designed to test JavaFX. You pass it the root node of your scene, then in these tests use commands, such as click, with either the id, or name of the item. It will then automatically control the mouse, interacting with the user interface.

\subsection{Integration tests}
\label{subsec:integration_tests}

In addition to unit testing, I performed integration tests that would test the interactions between the various modules in order to check that they are working correctly. Such as the live updater and its corresponding calls to the other modules. This also included testing that the user interface would correctly interact with the model and its various modules correctly.

\subsection{Manual testing}
\label{subsec:mamual_tests}

Where such tests needed a human eye, such as the design and drawing of items, and other small interactions that could not be automated. I manually over saw theses test. However, the majority of manual testing, were small things that would not greatly effect the application.
