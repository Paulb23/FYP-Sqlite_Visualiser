%
% The MIT License (MIT)
%
% Copyright (c) 2016 Paul Batty
%
% Permission is hereby granted, free of charge, to any person obtaining a copy
% of this software and associated documentation files (the "Software"), to deal
% in the Software without restriction, including without limitation the rights
% to use, copy, modify, merge, publish, distribute, sublicense, and/or sell
% copies of the Software, and to permit persons to whom the Software is
% furnished to do so, subject to the following conditions:
%
% The above copyright notice and this permission notice shall be included in
% all copies or substantial portions of the Software.
%
% THE SOFTWARE IS PROVIDED "AS IS", WITHOUT WARRANTY OF ANY KIND, EXPRESS OR
% IMPLIED, INCLUDING BUT NOT LIMITED TO THE WARRANTIES OF MERCHANTABILITY,
% FITNESS FOR A PARTICULAR PURPOSE AND NONINFRINGEMENT. IN NO EVENT SHALL THE
% AUTHORS OR COPYRIGHT HOLDERS BE LIABLE FOR ANY CLAIM, DAMAGES OR OTHER
% LIABILITY, WHETHER IN AN ACTION OF CONTRACT, TORT OR OTHERWISE, ARISING FROM,
% OUT OF OR IN CONNECTION WITH THE SOFTWARE OR THE USE OR OTHER DEALINGS IN
% THE SOFTWARE.
%

\section{Testing}
\label{sec:testing}

As mentioned at the start of this paper, and during the design section. One of the aims is to make sure that the tool could be relied on. In order to accomplish this, a variety of testing methods and tools were used, this section will go over theses.

\subsection{Test data}
\label{subsec:test_data}

In order to assure that the program ran correctly under a variety of circumstances, a variety of different sized databases were used to test. The smallest made up of one table and page, and the largest twelve tables and 1066 pages. The smaller and medium sized  databases were custom made for this project, while the larger one was taken from \cite{largedatabase} using real word data.
\\\\
As stated at the start of this paper SQLite is intended to be used as a file format rather than a complete database system. Therefore testing the application on a multi-gigabyte file would be unnecessary. This can be assumed as any databases of that size would be better off in a different system \citep{sqlite}. Although where this is out of the developers control or there are no alternatives to their situation would be classed as an edge case.
 
\subsection{Unit tests}
\label{subsec:unit_tests}

Throughout the implementation stage the process kept close to test driven development, and as such a  lot of unit tests have been written utilising the JUnit framework by \cite{junit} . In total there are around 139 unit tests. All passing. The unit tests are written to use mocks where dependences are needed allowing the tests to make sure the application is working as intended.
\\\\
The unit tests themselves, attempt to test all the possible action that could be performed to each exposed method in an attempt to make sure that each part of the program is operating as expected. Apart from the very few edge cases. However, some parts of the application are better tested then others due to the complex nature of some modules.
\\\\
In order to test the user interface a test framework that works alongside JUnit called TestFX \citep{test_fx} was used, it is specifically designed to test JavaFX. It takes the root node of the scene to test. Then in the tests command methods are used such as click, move to and hover with either the identifier, or name of the item. It will then automatically control the mouse, interacting with the user interface.
\\\\
TestFx works very well for testing the navigation methods around the user interface and making sure that the user can get from one part of the application to another. With that said for more complex and fine grained interaction such as the scroll pane within the visualiser, and the opening of titled panes in the log, were hard to test effectively, in such cases manual testing had to be used.

\subsection{Integration tests}
\label{subsec:integration_tests}

In addition to unit testing, integration tests were used to test the interactions between the various modules in order to check that they are working correctly. Such as the live updater and its corresponding calls to the other modules. This also included testing that the user interface would correctly interact with the model and its various modules correctly.
\\\\
Integration tests where combined with the unit tests, For example when the TestFX tests clicks on the various buttons in the menu bar the signals are still sent to the model. This  was used to effectively to make sure that the various elements with the program where working as expected.

\subsection{Manual testing}
\label{subsec:mamual_tests}

Where such tests needed a human eye, such as the design and drawing of items, and other small interactions that could not be automated. Manually overseeing theses items and how they interacted had to be used. In addition to the testing of visual elements, other manual tests include how easy the interaction with the application, for example opening a file. This was used as in general the automated tests do not take into account how the application is displayed and interactive by an actual user.
