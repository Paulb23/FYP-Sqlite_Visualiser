\section*{Introduction}
\label{sec:introduction}

Having studied Databases in my previous year, including SQLite. I remember being amazed that it could do so much without the need to configure, manipulate, or go through a long winded install process. It was so simple and flexible, anyone could use it. In fact SQLite takes pride that it is probably one of the most widely deployed database engines. And one of the top five most deployed software modules. Alongside zlib, libpng and libjpg. It finds itself inside all of the top browsers (Firefox, Google chrome and possibly Edge), Operating systems (Windows 10, IOS and embed OS's) and in the most unexpected places such as aircraft.
\\\\
After doing some reading and looking at the SQLite claim to fame. I began to take a closer look at its systems. What makes it so flexible, fast and simple, to use. My main focus however was on the file format that it uses to store the entire database. This included many long nights looking at how it was put together. How to traverse it. And why it is the way it is. I also looked at the available tools that are available for SQLite. This is covered in the first section of this paper.
\\\\
Understanding the file format was just the first stepping stone, as I then undertook a journey to build a tool that could traverse and read the file. While recording every operation that was and ever will be performed to it. This is covered in the second and third chapters.
\\\\
While building by tool. I kept two things in mind. How it operated form a users perspective and the best ways to break it. This was to ensure that the tool was open to everyone, and would not fall down and crumble, at the first chance it got. This is covered in sections four and five. 
\\\\
Once the tool became well developed. I started looking towards the future of the tool. What could be added to make it ever more useful for developers, researchers and anyone else that is using SQLite systems. This is covered in the final sections, six and seven.
\\\\
The main aim of this paper is to help you understand the SQLite file format and systems. While providing a useful tool that can help debug, manipulate and record your own SQLite databases, without the need for a hex editor.