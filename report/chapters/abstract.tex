\section*{\centering Abstract}

This paper presents a tool designed to visualise the internal workings of a SQLite database file. Starting with the history of SQLite, its systems and file format. Finding the file is a series of fixed sized chunks / pages, and each page is a node in a much larger B-Tree structure. Secondly, constructing a model-view-controller style application that can then parse, and present this data in real time, while other systems access the file. Thirdly, how using TestFX and JUnit have helped build a robust application. Lastly, how the tool could be improved by polishing up the user interface, with customisation and other minor interactions. The overall system could be improved with increased performance and adding some much needed features. On top of this, a future project could look at the changes made to the database and turn them back into the original SQL queries. 

\vspace{1.5cm}

\textbf{Keywords:} SQLite, databases, SQL, Java, JUnit, TestFx, JavaFX, B-Trees, Merkle trees, varints, Real time updates, User interface design