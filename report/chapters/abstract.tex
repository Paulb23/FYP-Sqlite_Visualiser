\section*{\centering Abstract}

This paper presents a tool designed to visualise the internal workings of a SQLite database file. We start by looking at the SQLite history, systems and file format. Finding the file is a series of fixed sized chunks / pages, and each page is a node in a much larger B-Tree structure. We then look at a model-view-controller style application that can then parse, and present this data in real time, while other systems access the file. To finish off we look at how using TestFX and JUnit have help build a robust application. Following on from that we see that the tool could be improved by polishing up the user interface, with customisation and other minor interactions. The system while complete could be improved with increased performance and adding some much needed features. Lastly we see a future project that could look at the changes made to the database and turn them back into the original SQL queries. 

\vspace{1.5cm}

\textbf{Keywords:} SQLite, databases, SQL, Java, JUnit, TestFx, JavaFX, B-Trees, Merkle trees, varints, Real time updates, User interface design