%
% The MIT License (MIT)
%
% Copyright (c) 2016 Paul Batty
%
% Permission is hereby granted, free of charge, to any person obtaining a copy
% of this software and associated documentation files (the "Software"), to deal
% in the Software without restriction, including without limitation the rights
% to use, copy, modify, merge, publish, distribute, sublicense, and/or sell
% copies of the Software, and to permit persons to whom the Software is
% furnished to do so, subject to the following conditions:
%
% The above copyright notice and this permission notice shall be included in
% all copies or substantial portions of the Software.
%
% THE SOFTWARE IS PROVIDED "AS IS", WITHOUT WARRANTY OF ANY KIND, EXPRESS OR
% IMPLIED, INCLUDING BUT NOT LIMITED TO THE WARRANTIES OF MERCHANTABILITY,
% FITNESS FOR A PARTICULAR PURPOSE AND NONINFRINGEMENT. IN NO EVENT SHALL THE
% AUTHORS OR COPYRIGHT HOLDERS BE LIABLE FOR ANY CLAIM, DAMAGES OR OTHER
% LIABILITY, WHETHER IN AN ACTION OF CONTRACT, TORT OR OTHERWISE, ARISING FROM,
% OUT OF OR IN CONNECTION WITH THE SOFTWARE OR THE USE OR OTHER DEALINGS IN
% THE SOFTWARE.
%

\section*{\centering Abstract}

This paper presents a tool designed to visualise the internal workings of a SQLite database file. Starting with the history of SQLite, its systems and file format. Finding the file is a series of fixed sized chunks / pages, and each page is a node in a much larger B-Tree structure. Secondly, constructing a model-view-controller style application that can then parse, and present this data in real time, while other systems access the file. Thirdly, how using TestFX and JUnit have helped build a robust application. Lastly, how the tool could be improved by polishing up the user interface, with customisation and other minor interactions. The overall system could be improved with increased performance and adding some much needed features. On top of this, a future project could look at the changes made to the database and turn them back into the original SQL queries. 

\vspace{1.5cm}

\textbf{Keywords:} SQLite, databases, SQL, Java, JUnit, TestFx, JavaFX, B-Trees, Merkle trees, varints, Real time updates, User interface design